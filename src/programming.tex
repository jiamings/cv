\section{Programming Experience}
\cvline{}{Proficient in C++, Python and Matlab. Capable of Java, \LaTeX, Julia, C\#, R, CUDA, Javascript, HTML/CSS, VHDL and Verilog. Some of my projects can be found on \href{http://jiamings.github.io/projects}{\color{blue} jiamings.github.io/projects}.}

\cventry{\footnotesize December 2014}{\href{http://jiamings.github.io/projects/decaf-video}{Video Classification with Visual and Audio Features}}{}{, Course Project.}{}{This project aims to do fast and scalable video sequence classification through deep feature extraction methods. We use \href{http://caffe.berkeleyvision.org}{\bfseries Caffe} for deep visual feature extraction.}{}

\cventry{\footnotesize May 2015}{\href{http://jiamings.github.io/projects/georun}{GeoRun - A Unity Game with Kinect Controls}}{}{, Course Project}{}{We developed GeoRun, which is a simplified Temple Run game developed with \href{https://unity3d.com/}{Unity} and Kinect SDK v1.8.}

\cventry{\footnotesize June 2015}{\href{http://jiamings.github.io/projects/tusk}{TUSK - Tsinghua University Search Kit}}{}{, Course Project}{}{A search engine over Tsinghua news and documents with auto-completion and voice search.}{}

\cventry{\footnotesize November 2015}{\href{http://jiamings.github.io/static/slides/EPOC.pdf}{EPOC - Emotion Personalized | Online Chat}}{}{, for \href{http://www.hackshanghai.com/}{\color{blue} HackShanghai}, China's largest hackathon.}{}{Modifying wallpapers and background music by mind, with the help of \href{https://emotiv.com/epoc.php}{\color{blue} Emotiv EPOC}.\\ \href{http://www.icshanghai.com/en/information/2015-11-10/40507.html}{\color{blue} Our project was reported by International Channel Shanghai.}}

% \subsection{Proficient}
% \cvline{\small C / C++ / CUDA / \LaTeX}{Implemented acceleration methods for link prediction (2000+ lines of code); implemented core algorithm for the Outstanding paper in ICM 2015 (800+ lines); implemented extensions to Caffe to create a convolutional neural network for multilabel image annotation.}
% \cvline{Python}{Developed Biopedia, a web service for the Bioinformatics group in Tsinghua, using the Flask framework and insights of Google Material Design. The service is currently deployed at \href{http://biopedia.bigdata-thu.org}{\color{blue} biopedia.bigdata-thu.org}.}
% \cvline{Java}{Developed a simple instant messaging application with Google Protocol Buffer during exchange in NTHU.}
% \cvline{Matlab}{Developed deep generative models at Duke for the AISTATS paper.}
% \subsection{Familiar (used in at least 1 course project)}
% \cvline{}{Bash, HTML, VHDL, Verilog, R, C\#.} 
