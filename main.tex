%% start of file `template.tex'.
%% Copyright 2006-2013 Xavier Danaux (xdanaux@gmail.com).
%
% This work may be distributed and/or modified under the
% conditions of the LaTeX Project Public License version 1.3c,
% available at http://www.latex-project.org/lppl/.


\documentclass[11pt,a4paper,sans]{moderncv}        

% modern themes
\moderncvstyle{classic}                           
\moderncvcolor{blue}                                

% character encoding
\usepackage[utf8]{inputenc}                       

\usepackage[scale=0.75]{geometry}


\usepackage{import}

\name{Jiaming}{Song}

\phone[mobile]{+86 18001585905}                   
\email{jiaming.tsong@gmail.com}                               
\homepage{www.github.com/jiamings}                         


\begin{document}
\maketitle
\section{Education}
\cventry{2012 -- 2016}{Tsinghua University}{Beijing, China}{}{}{}
\cvline{}{B.Eng. (Expected) in Computer Science and Technology}
\cvline{}{{\bfseries Major GPA}: 92.50/100. {\bfseries Major Rank}: 3/137. }
\cventry{July 2014}{National Tsing Hua University}{Hsinchu, Taiwan}{}{}{}
\cvline{}{Exchange student, Electrical Engineering and Computer Science}

\section{Research Experience}
\cventry{\footnotesize November 2014 -- Current}{State Key Lab of Intelligent Tech. and Systems (TNList)}{Tsinghua University}{}{Advisor: Prof. Jun Zhu}{Further explored more acceleration method for link prediction problems. Proposed an efficient method that would train on a network with over 3 million nodes, which is over ten thousand times increase over original methods. Preparing to submit to IEEE Trans. PAMI. very soon.}{}

\cventry{\footnotesize July 2014 -- October 2014}{Visual Computing Group}{Microsoft Research Asia}{}{Advisor: Jingdong Wang}{Worked on classification and detection algorithms using deep learning methods; studied and modified Caffe, and open-source deep learning framework in C++ and CUDA; developed a convolutional neural network for multiple label image annotation which achieved state-of-the-art precision results.}{}

\cventry{\footnotesize October 2013 -- June 2014}{State Key Lab of Intelligent Tech. and Systems (TNList)}{Tsinghua University}{}{Advisor: Prof. Jun Zhu}{Explored and implemented stochastic variational inference methods for max-margin latent feature relational model, which is used for link prediction. Implemented a Gibbs sampling benchmark algorithm for Scalable Inference for Logistic Normal Topic Models (accepted by NIPS 2013).}{}

\section{Honors and Awards}
\cventry{\small April 2015}{Outstanding Winner}{2015 Interdisciplinary Contest in Modeling}{issued by the Consortium for Mathematics and Its Applications (COMAP)}{}{Highest award (9 out of 2317) of the contest. Wrote a paper which models organizational churn using Bayesian methods and network science. See \url{github.com/jiamings/icm2015} for more details. }
\cventry{\small October 2014}{Outstanding Undergraduate}{}{issued by the China Computer Federation (CCF)}{}{Only 2 students in Tsinghua, and 100 in China are awarded. Attended China National Computer Congress, where we received the award and had the pleasure to meet Alexander Wolf(President of the ACM) and Ivan Sutherland(Turing Award 1988).}{}
\cventry{\footnotesize September 2014}{Tsinghua-Hengda Scholarship}{}{issued by Evergrande Group}{}{}{}
\cventry{\small May 2014}{Spark Program for Technological Innovation}{}{Tsinghua University}{}{Among top 50/3000 students for achievements in scientific and technological innovations.}{}
\cventry{\footnotesize December 2013}{Zhong Shimo Scholarship}{}{issued by Dept. of Computer Science and Technology}{}{No.1 scholarship in the Dept. for academic achievements, social activities, and charity work.}{}

\section{Programming Experience}
\subsection{Proficient}
\cvline{\small C / C++ / CUDA / \LaTeX}{Implemented acceleration methods for link prediction(2000+ lines of code); implemented core algorithm for the Outstanding paper in ICM 2015(800+ lines); implemented extensions to Caffe to create a convolutional neural network for multilabel image annotation.}
\cvline{Python}{Developed Biopedia, a web service for the Bioinformatics group in Tsinghua, using the Flask framework and insights of Google Material Design. The service is currently deployed at \url{biopedia.bigdata-thu.org}.}
\cvline{Java}{Developed a simple instant messaging application during exchange in NTHU.}
\subsection{Familiar}
\cvline{}{Matlab, Bash, HTML, VHDL, Verilog, R, C\#.} 

\section{English Proficiency}
\cvline{TOEFL}{Total: 111 (Reading: 30; Writing: 29; Listening: 27; Speaking: 26).}
\cvline{GRE}{Verbal: 159/170; Quantitative: 170; Analytical Writing: 3.5.}

\section{Activities and Societies}
\cventry{\footnotesize March 2013 -- Current}{Association for Student International Communication}{}{Tsinghua University}{}{One of the top student associations in Tsinghua devoted to projects that serve to expand the international vision for Tsinghua students. ASIC has won the Top 10 associations in Tsinghua for 5 years straight, while being the first in 3 years.}{}
\cventry{\small September 2013 -- Current}{Badminton Team}{}{Dept. of Computer Science and Technology}{}{}{}
\cventry{\small August 2014}{Initiating Mutual Understanding through Student Exchange}{Vice President}{Tsinghua U, Peking U, and Harvard U.}{}{Organized the event planning and design process for IMUSE, a 8-day forum where students from China and the US. share their life stories and thoughts, and experience life together.}{}
\cventry{\small August 2013}{Building Bridges Charity Project}{Coordinator}{Tsinghua University}{}{Spend one week teaching high school students in Lishui, Zhejiang with students from Tsinghua U., Peking U. Yale U, etc. Coordinated the team of Tsinghua students.}{}
\end{document}


%% end of file `template.tex'.
